\documentclass{article}
\usepackage{xspace}
\usepackage[breaklinks=true]{hyperref}
\usepackage{listings}


\title{Git and \LaTeX}
\author{Seamus Bradley\\Charlie Tanksley}
\date{\today}

\begin{document}
\lstset{language=[LaTeX]TeX,breaklines=true,frame=single,frameround=tttt}
  \maketitle
  \tableofcontents

If you are using \LaTeX, you should be using a version control system 
or VCS. In this post I’m going to explain what a VCS is and why you 
should use one. In a later post I’ll explain how I use the particular 
version control system I use, \href{http://git-scm.com}{git}. Which 
version control system you use is not terribly important. But I think 
using a VCS is.

\section{What is a VCS?}

If you’ve ever used the ‘Track Changes’ feature in Microsoft Word, 
you’ve used \emph{revision control}. A \emph{version control system} 
is a program that basically tracks the revisions you make to your 
document. The term ‘version control’ comes from the idea that this 
software tracks versions of your documents. Apparently that is the way 
old VCSs worked; now, 
\href{http://www.joelonsoftware.com/items/2010/03/17.html}{from what I 
  have read}, apparently the VCS you will use tracks sets of changes 
to your document, not revisions. From my perspective, that is a 
distinction without a difference, though under the hood I’m guessing 
it makes a substantial difference.

So a version control system is a way for you, the author of a paper, 
to track the changes you have made to that document. But there is a 
difference between ‘Track Changes’ and a VCS—with a VCS, you are in 
control: you save \textbf{chunks} of changes, not individual changes 
(though an individual change is the smallest chunk!). What is more, 
you manually append a message to each hunk of changes. These messages, 
and the discrete units of change, are two of the things that make a 
VCS so powerful.

\section{Basic mechanics}

You are writing a paper, and you are doing so in \LaTeX with your 
favorite editor.  Suppose the paper is \verb!paper.tex!, and suppose 
you are at the point where you are revising the paper—you have a draft 
and you need to rework numerous sections. If you are using a VCS 
(which you have already told to watch \verb!paper.tex!) your basic 
workflow will look something like this:

\begin{enumerate}
\item
  Write some. Maybe you delete half of a section and rewrite it.
\item
  \emph{Commit} your changes. That is, you have already told your VCS 
  to keep its eyes on \verb!paper.tex!, now you are telling the VCS 
  about the changes you’ve made. Your VCS will know exactly what words 
  you changed, what you added, and what you deleted; when you commit 
  your changes you are telling the VCS to (a) save the hunk of changes 
  you just made as one \textbf{unit} and (b) attach a description of 
  those changes that you provide.
\item
  Write more.
\item
  Commit those changes.
\end{enumerate}
Most of your time will be spent like that. You will change section 1, 
then tell your VCS to save those changes along with the comment 
‘changed section 1 to narrow the focus’, for example.

\section{Slightly more advanced mechanics}

If that was \emph{all} a VCS did it might be valuable; but a VCS can 
do much more.

Suppose you decide that you want to try out some pretty radical 
changes to the paper—maybe you move section 7 to the front, cut out 
sections 2 and 6, and write three new sections at the end. Plus you 
change the definition of some key term (which you repeat throughout).

These major rewrites are common. They are also scary. What if you make 
all those changes and the new version turns out worse than the current 
version? How would you go back to where you started?

If you are using a VCS, that mistake isn’t quite so terrible. It isn’t 
terrible because all you have to do is look at the history of your 
document, find the old version (this isn’t very hard, and you can make 
it even easier if you have a proper workflow), and issue a single 
command to revert your paper to that old version.

(I realize you could do something like this without a VCS—you could 
save a copy of the paper \verb!good_version_from_december_7.tex! or 
whatever. I’ll talk more about this below.)

Or suppose that you realize that while the new version isn’t better 
than the old version, the new definition you introduced for that term 
(and changed throughout the document) is much better. If you know how 
to use your VCS, you can throw away all the changes you made to the 
document \emph{except} that changed definition. That change you can 
keep. So you end up with the old version plus the new definition 
throughout.

\section{But why use a VCS?}

People who don’t use a VCS likely have one of two approaches to paper 
versions. One is the \verb!paper_1.tex!, \verb!paper_2.tex!, etc. 
approach. With this approach, you create a new draft at every 
significant milestone or when you want to try something new. The other 
approach is to have one version of the paper and just plow ahead; you 
commit to changes or to figuring out how to get things back if you 
make a mistake.

(There is a new version of this second approach, which is to put your 
paper in a Dropbox folder. If you do this, then every time you save 
the file you will replace the old version, \textbf{but} Dropbox will 
keep that old revision for some amount of time. So you can use Dropbox 
to go back in time to an older version (Mac OS X’s Time Machine does 
the same thing). This is kind of like a Track-Changes-in-chunks 
feature, where the chunks are demarcated by your saving operations. 
But I don’t think it is as good as using a VCS. More on this to 
follow.)

These aren’t terrible strategies at all. And maybe you don’t want to 
(or can’t) be bothered with learning yet another new piece of software 
to track your changes, so maybe you want to stick with one of these 
methods. That’s fine; you should stick with your current strategy.

But there is a better way: use a VCS. I think there are \textbf{four} 
advantages to using a VCS:

\begin{enumerate}
\item
  It is elegant.
\item
  It allows you to experiment for low cost.
\item
  It gives you control over your version history. And control is a 
  good thing (you use \LaTeX don’t you?  Isn’t that because it gives 
  you control over your work?).
\item
  It gives you an actual \emph{history} for your document.
\end{enumerate}

\section{A VCS is elegant}

Above I noted that there are two common strategies for keeping up with 
drafts. The first is the save-multiple-drafts method. Everyone knows 
this method is no good. You usually don’t have a clue about what has 
changed from draft to draft. There is no way to use the information in 
the old draft unless you just go back to it and throw out anything new 
you’ve done. This kind of system is like a safety net of sorts. But 
when I was using this system I knew I’d never go back to an old draft. 
So they were more or less pointless.

A VCS keeps one copy of your file. It has its own mechanism for 
keeping track of older versions. So you have one \verb!paper.tex!  
file. If you decide you want to go back to a version from two months 
ago, you issue a command and the contents of \verb!paper.tex! change 
to that old version. The new one is still there—a simple command 
brings you back. So \verb!paper.tex! is now slightly magical: it can 
turn into any document in its history at your command.

When you make a commit (tell the VCS about some changes), you write a 
commit message. You could make these as uninformative as you like (my 
first few months of messages were pretty weak). So if you change the 
definition of ‘four-dimensionalism’ throughout you could have a commit 
message that reads \verb!made some changes! or \verb!changed the 
definition of four-dimensionalism throughout!. If you go with the 
latter, then when you look at your history you can see exactly what 
you did.

Most VCSs will let you tag a version of a file. So you could say that 
the version of \verb!paper.tex! from 8am on November 2010 is v1.0. 
When you do that, you should have the option to write a detailed 
message about this version. So you could say what you liked about this 
version but what you are thinking about changing.  Then when you look 
at your history you can see how each version differs.

\section{Experimentation}

I abandoned the save-multiple-drafts method because it was pointless 
for me. Never once did I go back to an older version. The problem was 
the opacity of versions: I had no clue what I had changed from one 
version to the next. The previous versions were dead to me. Instead, I 
just plowed ahead with one version, making changes as I saw fit.

The problem with this approach is fear. If you have a decent draft, 
you don’t want to ruin it by taking a risk. So that good idea goes 
unexplored. If I strike off in a new direction and change my mind, I 
can’t go back. So I don’t want to experiment. I’ll take a new 
direction, but only if I’m pretty sure about it.

With a VCS, I can stick with the barge-ahead method, but I know I can 
always go back to wherever I need. It seems to me to be the best of 
both worlds. If I want to try a completely new structure, I can do 
that without fear. With a simple command I can switch back and forth 
between versions.

(Thanks do 
\href{http://philosophy.csusb.edu/~dotto/me/Home.html}{Darcy Otto} for 
talking about these ideas with me.)

\section{Control}

Using a VCS is one extra step in the writing process. At regular 
intervals you have to explicitly put a hunk of changes in your 
repository and write a commit message. At first this might seem like a 
nuisance. And for some it might be. But I like it. I like that it 
gives me a chance to plan what I want do to do next, to do just that 
much, and then to explain what I did. It gives my writing more 
purpose. I’m not sitting down to bang out ideas and words—I’m sitting 
down to explain how this conclusion is a problem for so-and-so. That 
direction helps me as a writer.

Dropbox and Word’s Track Changes will keep track of versions of your 
paper. But you have very little control over what those versions look 
like. If you save often and reflexively (or automatically), then the 
difference between two Dropbox versions might be a single sentence. 
And there is no way to know what that difference is (well, you could 
save each version separately and use a diff tool to see, but that is a 
lot of work!). With a VCS, you determine how many, and which, changes 
are worth marking as a single hunk of work. Sometimes this is a 
burden. But usually it isn’t a very big deal. And the payoff is big: 
the different versions are meaningful, not random. They are in your 
control.

\section{History}

Suppose you wrote a draft of a paper and put it aside for a few weeks 
or months (perhaps because you got busy, perhaps to look at it with 
fresh eyes). How do you get reacquainted with the project when you 
pick it back up? You could just get cracking. Or you could read the 
whole paper. If you are using a VCS, you could view the log of commit 
messages you made. Though it won’t always do the trick, this can 
really put you back in touch with the paper. You don’t just see what 
you \textbf{have}, you see what you \textbf{had} and why you changed 
it to what you have now. You can see that you were spending lots of 
time on section 7, that you had doubts about a given argument, or 
whatever. (Of course, you have to have good commit messages for this 
to work!)

I use my VCS, git, for this all the time. I’m fond of a program called 
\href{http://jonas.nitro.dk/tig/}{Tig} that lets me view a graphical 
representation of my git repository in a terminal (there are other GUI 
alternatives). So I run \verb!tig --all! in a terminal and can very 
quickly see where I’ve been working recently.  This is something like 
a high-tech version of reading the last paragraph you wrote yesterday 
before you start writing today. Only this version is quite helpful 
when you are revising.

You could also use this for a longer term view of the project.  
\href{http://www.travisswicegood.com/}{Travis Swicegood} suggested the 
following to me (via Twitter):

\begin{quote}
Use it for review later. It gives insight into the 
creative/intellectual history of the work.

\end{quote}
This seems like a good idea to me, if you are interested in such 
things. If you kept good commit messages, you might find some new 
research ideas by going back through your history.

\section{Conclusion}

I don’t think a VCS is for everyone. It does take some work, and that 
extra work might not be justified for you. But I’ll just say that I 
don’t know anyone who has started using a good VCS (e.g, 
\href{http://mercurial.selenic.com/}{Mercurial} or 
\href{http://git-scm.com/}{git}) and regretted it. Most of us can’t 
live without it and can’t imagine going back. Just like most of us 
\LaTeX users can’t imagine going back to Word.

\end{document}
